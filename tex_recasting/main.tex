\documentclass[11pt]{cernrep}
\usepackage{graphicx,epsfig, url}
\usepackage[utf8]{inputenc}
\bibliographystyle{lesHouches}
\begin{document}

\setlength\parindent{0pt}

\title{Recasting activities at LH2017}

\author{L.~Perrozzi$^1$, Fabio~Maltoni, Sabine~Kraml, Gabriel~Facini, David~Grellscheid, Sezen~Sekmen,
  Jonathan~Butterworth, Nishita~Desai, Andy~Buckley$^{AB}$, Benjamin~Fuks, Eric~Conte, Peter~Richardson,
  Olivier~Mattelaer, Pasquale~Musella, Alexandra~Oliveira~Carvalho, Ursula~Laa, Kristin~Lohwasser,
  ???~Thrynova, Efe~Yazgan, Philippe~Gras$^{20}$, Sylvain~Fichet}

\institute{
$^1$ IPA at ETH Zurich, Switzerland
\\
$^2$ \dots\\
$^{AB}$ School of Physics \& Astronomy, University of Glasgow, UK\\
$^{20}$ IRFU, CEA, Universit\'e Paris-Saclay, Gif-sur-Yvette
}

\maketitle

\begin{abstract}
The scope of this report, therefore, is to document the advancements obtained so far on the recasting
activities and attempt a first benchmark to compare different tools to reproduce several ATLAS and
CMS analysis results.
\end{abstract}

\section{Introduction}

Searches for new physics constitute a basic ingredient of the LHC physics program.
Their large number and variety pose severe challenges to both the experimental and theory communities.
In fact, hundreds of searches are performed by different collaborations, several final states are used,
new ideas on how to probe new models and non-trivial signatures and improve the sensitivity of existing searches constantly emerge.
The ultimate goal of this effort is to discover new physics if such
exists within the reach of the LHC, and to test the widest possible range of hypothetical new physics models.

A typical analysis defines quantities that aid in classifying the event as signal or background: for example
the properties of analysis objects such as jets, electrons, muons, etc., or global event variables
such as object multiplicities, transverse momenta, transverse masses, etc.
An analysis can be very complex and feature many intricate definitions of object and event
variables, some of which cannot be expressed in closed algebraic form and must be defined
algorithmically. This complexity renders the task of visualizing, understanding, developing and
interpreting analyses increasingly challenging.

One obvious way to cope with the complexity is to devise ways to enforce absolute clarity in the description of analyses.
A discussion was started in the Les Houches PhysTeV workshop in 2011, and continued
thereafter within a wider group of LHC physicists, in order to determine what information is
crucial for describing an analysis. The outcome of this discussion was reported in the
''Recommendations for Presentation of LHC Results''~\cite{Brooijmans:2012yi,Kraml:2012sg,Brooijmans:2016vro}, and has been embraced by many LHC physicists.

The current practice in our community is to write an analysis in non-public computer codes,
which often rely on event objects specific to the experimental collaboration in question,
and then make public a description of the analysis via journal publications or other documents.

These efforts, which merit our great appreciation,
have significantly increased the scientific value of many important experimental results.

There is significant precedent for the effectiveness of such community standards. Several
accords have been established to standardize the communication of physics modeling information,
notably the Les Houches Event Accord (LHE)~\cite{Boos:2001cv,Alwall:2006yp} and the SUSY Les Houches Accord (SLHA)~\cite{Allanach:2008qq,Skands:2003cj}.
These, respectively, standardize the description of hard-process
particles in simulated collision events, and the details of all the parameters that define a BSM model point.
Both accords are widely used in high-energy physics and have greatly helped to
simplify and make more efficient the communication between physicists.
In recent years, the need for a standardized format was also underscored, i.e. an “analysis description
accord”, capable of describing the contents of an analysis in an unambiguous way, which can
be fully exploited by the whole particle physics community.  The accord must be capable of
describing all object and event selections, as well as quantities such as efficiencies, analytic and
algorithmic observables, and advanced multivariate selections.
In the Les Houches PhysTeV workshop in 2015, a dedicated discussion has been initiated
on how such an accord can be realized~\cite{Brooijmans:2016vro}.
The important motivations and use-cases for a standard analysis description accord were envisaged as:
analysis preservation, Analysis design, analysis review and communication, interpretation studies and analysis reimplementation,easier comparison of analyses.
In addition, there are several desirable features which would further improve the utility
of the accord, which, however, may be nontrivial to simultaneously fulfill.  
The features required for the success of such an accord would be divided into basic requirements (public availability, completeness, longevity, correctness and validatability) and desirable features (human readability and writeability, self-contained, language independence, framework independence, support for combination of analyses).
Other proposals in this direction have been realized~\cite{Collaboration:2242860,recasting_atlas}.

Several discussions and progresses have been made, but the proposal is not yet final and has not been widely adopted yet.

% In the following sections we shall detail the use cases
% and design requirements of such an accord, and the general pros and cons of several approaches.
The scope of this report, is to document the advancements obtained so far
on the recasting activities and attempt a first benchmark to compare different tools
to reproduce several ATLAS and CMS analyses' results provided through HEPDATA~\cite{Maguire:2017ypu}.
To trust the reliability of this recasting approach, we realize feasibility studies of the implementation and portability of complicated MVA techniques (BDT, NN,…) into the recasting frameworks featuring different approaches like detector smearing (DELPHES~\cite{deFavereau:2013fsa}) results and simple object smearing.

Further improvements in the presentation of the results and their recastability will be to provide correlations signal systematics, as well as the possibility to provide a few key observables unfolded. Finally, it would be interesting in the future to try using particle-level measurements to constrain models.

% \section{Formats}
% Object efficiency tables : which format (HEPDATA?)

\section{Benchmarking/Comparisons}

\begin{itemize}
  \item Implementation of analyses of increasing complexity in the Analysis Description Format (LHADA Proposal) and in (BSM) Rivet and their comparison.
  \item Choose an analysis of ATLAS or CMS which has cutflow and detector effects provided in some form, and possibly is already been implemented in the recasting codes CheckMate/MadAnalysis/Rivet/ATOM.
  \item Implement the same analysis in LHADA and then use the dedicated parsers to provide the analysis for the recasting codes.
  \item Reproduce the NP interpretation of the original paper (=validation implementation).
  \item Recast the analysis for an other new physics model and compare the results.
  \item Go to point one and choose a more complicated analysis…
\end{itemize}
it would be interesting to see how Delphes performance looks without analysis-specific cards, since a lot of people (outside the “big” recasting groups) are using it that way.

\section{How to validate the analyses}
\begin{figure}
\begin{center}
\includegraphics[width=0.5\textwidth]{figures/lhada_benchmarking_excersise.png}
 \caption{Search reach for the $\mu \gamma {\not\!\!E}_{T}$ signal
(as defined in the
   text) for
   300 fb$^{-1}$ integrated luminosity  at the LHC.
}
\label{search}
\end{center}
\end{figure}


\section{The analysis frameworks and tools}
In this section we describe the analysis frameworks and tools used for the comparison and benchmarking
% \subsection{ATOM}
% Atom is a general purpose framework for reinterpreting existing experimental analyses and designing new ones. Originally started as a fork of Rivet, Atom includes the possibility of describing detector effects by using transfer functions from truth-level objects to detector objects. In addition to providing a recasting framework, Atom emphasizes checking the validity of the extrapolations intrinsic in recasting a result from one BSM model to another. This is achieved via a flexible system warning the user of various conditions invalidating the results. On top of providing a database of existing BSM ATLAS and CMS analyses, LHC run I and II detector descriptions, while still being fully compatible with all Rivet analyses, Atom delivers tools for easily implementing new analyses, including an automatic validation system.

\subsection{CheckMate}
CheckMATE takes simulated event files in .hep or .hepmc for any model as input and simply returns if the underlying model is "excluded" or "allowed" after performing a detector simulation and testing various implemented analyses. The embedded AnalysisManager makes it easy to add current and prospective future LHC results from ATLAS and CMS which have not yet been implemented. Detector effects are considered by Delphes extended by tuned efficiency functions for lepton reconstruction and flavour tagging. The soon-to-be published version 2.0 adds the possibility of using Pythia8 to generate SUSY events on-the-fly or to shower provided .lhe files for any model. Currently, the collaboration is working on an extension to enable the on-the-fly simulation of events for any model.


\subsection{MadAnalysis}
MadAnalysis 5 is a generic user-friendly framework for phenomenological investigations at particle colliders, i.e. to perform physics analyses of Monte Carlo event files. Its Public Analysis Database (PAD)comprises a growing collection of LHC analyses, which have been implemented in the MadAnalysis 5 framework for the purpose of recasting. Delphes3 is used for the detector simulation. For each implemented analysis, a detailed validation note is provided. The PAD follows an open-source policy; contributed codes are published and citable via Inspire. The framework is currently being extended to provide a full recast chain, from Madgraph to limit setting.

\subsection{Rivet}
Originally developed as a toolkit for the validation of Monte Carlo event generators, Rivet (Robust Independent Validation of Experiment and Theory) has become a standard for documenting [unfolded] SM measurements. The LHC experiment top and Higgs groups are also increasingly providing Rivet routines for their analyses. Rivet analyses are written in a user-friendly subset of C++11, and are picked up at runtime as "plugin libraries"; they can be executed on an event stream either through a Python script interface, or by direct code interfacing to a C++ API.
The original SM-focused requirement of unfolded observables made Rivet inappropriate for BSM searches (other than those using just jets and MET) until the addition of detector-smearing/efficiency machinery in Rivet 2.5.0. This detector machinery provides equivalent efficiency effects to a Delphes-type simulation, and imitates the less important kinematic smearing of physics objects to within a few percent. A novel feature is that the Rivet detector implementation allows jet algorithms, lepton and b-tagging operating points, full-detailed object isolation algorithms, and resolutions/efficiencies to be specific to each analysis's procedure and event-selection. This allows more accurate detector modelling and more robust analysis preservation than "global" detector simulations, hence addresses some experiment concerns re. requests for "official fast-sim" tools. The aim is to encourage Rivet code provision direct from BSM data analysers, as is already the case for SM results: additional tools to assist BSM analysis implementation are being added on request.

\subsection{Generic Analysis Description Proposal}
Brief description of the Generic Analysis Description Proposal

\section{Analyses benchmarking, comparisons and results}

\subsection{arxiv:1605.03814 - Jets+MET - ATLAS - 13 TeV}
Brief description of the ATLAS analysis Jets+MET at 13 TeV (arxiv:1605.03814).
Results are reported in table~\ref{tab:1605.03814}.

\begin{table*}[htbp]
\tiny
	\centering
		\begin{tabular}{ | l || l | l | l || l | l | l || l | }
\hline
% Rivet, MadAnalysis 5 and CheckMATE2 benchmarking results & \  & \  & \  & \  & \  & \  & \  & \  \\ \hline
% arxiv:1605.03814 - Jets+MET - ATLAS - 13 TeV & \  & \  & \  & \  & \  & \  & \  & \  \\ \hline
% Gluino mass 1600 - N1 mass 0 & \  & \  & \  & \  & \  & \  & \  & \  \\ \hline
% Rivet default (with efficiencies and smearing); MadAnalysis5 default; CheckMATE2 default & \  & \  & \  & \  & \  & \  & \  & \  \\ \hline
 % & \  & \  & \  & \  & \  & \  & \  & \  \\ \hline
                  &  \multicolumn{3}{c||}{\bf Rivet} & \multicolumn{3}{c||}{\bf MadAnalysis5} &   {\bf CheckMATE}   \\ \hline

Description       & \#evt & tot.eff & rel.eff & \#evt & tot.eff & rel.eff &   tot.eff   \\ \hline \hline
{\bf 2jl cut-flow}                  & 31250 & 1 & - & 31250 & 1 & - & \   \\ \hline
Pre-sel+MET+pT1   & 28592 & 0.91 & 0.91 & 28626 & 0.92 & 0.92 & \   \\ \hline
Njet              & 28592 & 0.91 & 1 & 28625 & 0.92 & 1 & \   \\ \hline
Dphi\_min(j,MET)   & 17297 & 0.55 & 0.6 & 17301 & 0.55 & 0.6 & \   \\ \hline
pT2               & 17067 & 0.55 & 0.99 & 17042 & 0.55 & 0.99 & \   \\ \hline
MET/sqrtHT        & 8900 & 0.28 & 0.52 & 8898 & 0.28 & 0.52 & \   \\ \hline
m\_eff(incl)       & 8896 & 0.28 & 1 & 8897 & 0.28 & 1 & \   \\ \hline
\hline
{\bf 2jm cut-flow} & 31250 & 1 & - & 32150 & 1 & - & 1  \\ \hline
Pre-sel+MET+pT1   & 28472 & 0.91 & 0.91 & 28478 & 0.91 & 0.91 & 0.91  \\ \hline
Njet              & 28472 & 0.91 & 1 & 28477 & 0.91 & 1 & 0.91  \\ \hline
Dphi\_min(j,MET)   & 22950 & 0.73 & 0.81 & 22889 & 0.73 & 0.8 & 0.73  \\ \hline
pT2               & 22950 & 0.73 & 1 & 22889 & 0.73 & 1 & 0.73  \\ \hline
MET/sqrtHT        & 10730 & 0.34 & 0.47 & 10710 & 0.34 & 0.47 & 0.33  \\ \hline
m\_eff(incl)       & 10630 & 0.34 & 0.99 & 10609 & 0.34 & 0.99 & 0.32  \\ \hline
\hline
{\bf 2jt cut-flow} & 31250 & 1 & - & 31250 & 1 & - & \   \\ \hline
Pre-sel+MET+pT1   & 28592 & 0.91 & 0.91 & 28626 & 0.92 & 0.92 & \   \\ \hline
Njet              & 28592 & 0.91 & 1 & 28625 & 0.92 & 1 & \   \\ \hline
Dphi\_min(j,MET)   & 17297 & 0.55 & 0.6 & 17301 & 0.55 & 0.6 & \   \\ \hline
pT2               & 17067 & 0.55 & 0.99 & 17042 & 0.55 & 0.99 & \   \\ \hline
MET/sqrtHT        & 5083 & 0.16 & 0.3 & 5098 & 0.16 & 0.3 & \   \\ \hline
Pass m\_eff(incl)  & 4861 & 0.16 & 0.96 & 4889 & 0.16 & 0.96 & \   \\ \hline
\hline
{\bf 4jt cut-flow} & 31250 & 1 & - & 31250 & 1 & - & 1  \\ \hline
Pre-sel+MET+pT1   & 28592 & 0.91 & 0.91 & 28626 & 0.92 & 0.92 & 0.91  \\ \hline
Njet              & 27322 & 0.87 & 0.96 & 27128 & 0.87 & 0.95 & 0.87  \\ \hline
Dphi\_min(j,MET)   & 18929 & 0.61 & 0.69 & 18829 & 0.6 & 0.69 & 0.6  \\ \hline
pT2               & 18715 & 0.6 & 0.99 & 18825 & 0.6 & 1 &     --       \\ \hline
pT4               & 16610 & 0.53 & 0.89 & 16430 & 0.53 & 0.87 & 0.52  \\ \hline
Aplanarity        & 11849 & 0.38 & 0.71 & 11395 & 0.36 & 0.69 & 0.36  \\ \hline
MET/m\_eff(Nj)     & 8334 & 0.27 & 0.7 & 7971 & 0.26 & 0.7 & 0.25  \\ \hline
m\_eff(incl)       & 7201 & 0.23 & 0.86 & 6972 & 0.22 & 0.87 & 0.21  \\ \hline
\hline
{\bf 5j cut-flow} & 31250 & 1 & - & 31250 & 1 & - & 1 \\ \hline
Pre-sel+MET+pT1   & 28592 & 0.91 & 0.91 & 28626 & 0.92 & 0.92 & 0.91 \\ \hline
Njet              & 21234 & 0.68 & 0.74 & 21185 & 0.68 & 0.74 & 0.68 \\ \hline
Dphi\_min(j,MET)   & 14294 & 0.46 & 0.67 & 14292 & 0.46 & 0.67 & 0.45 \\ \hline
pT2               & 14146 & 0.45 & 0.99 & 14289 & 0.46 & 1 &    --       \\ \hline
pT4               & 13229 & 0.42 & 0.94 & 13228 & 0.42 & 0.93 & 0.42 \\ \hline
Aplanarity        & 9836 & 0.31 & 0.74 & 9576 & 0.31 & 0.72 & 0.3 \\ \hline
MET/m\_eff(Nj)     & 4643 & 0.15 & 0.47 & 4506 & 0.14 & 0.47 & 0.13 \\ \hline
m\_eff(incl)       & 4620 & 0.15 & 1 & 4476 & 0.14 & 0.99 & 0.13 \\ \hline
\hline
{\bf 6jm cut-flow} & 31250 & 1 & - & 31250 & 1 & - & 1  \\ \hline
Pre-sel+MET+pT1   & 28592 & 0.91 & 0.91 & 28626 & 0.92 & 0.92 & 0.91  \\ \hline
Njet              & 13235 & 0.42 & 0.46 & 13236 & 0.42 & 0.46 & 0.41  \\ \hline
Dphi\_min(j,MET)   & 8520 & 0.27 & 0.64 & 8553 & 0.27 & 0.65 & 0.26  \\ \hline
pT2               & 8436 & 0.27 & 0.99 & 8551 & 0.27 & 1 &    --        \\ \hline
pT4               & 8135 & 0.26 & 0.96 & 8217 & 0.26 & 0.96 & 0.25  \\ \hline
Aplanarity        & 6365 & 0.2 & 0.78 & 6307 & 0.2 & 0.77 & 0.19  \\ \hline
MET/m\_eff(Nj)     & 2675 & 0.09 & 0.42 & 2665 & 0.09 & 0.42 & 0.08  \\ \hline
m\_eff(incl)       & 2670 & 0.09 & 1 & 2656 & 0.08 & 1 & 0.08  \\ \hline
\hline
{\bf 6jt cut-flow} & 31250 & 1 & - & 31250 & 1 & - & \   \\ \hline
Pre-sel+MET+pT1   & 28592 & 0.91 & 0.91 & 28626 & 0.92 & 0.92 & \   \\ \hline
Njet              & 13235 & 0.42 & 0.46 & 13236 & 0.42 & 0.46 & \   \\ \hline
Dphi\_min(j,MET)   & 8520 & 0.27 & 0.64 & 8553 & 0.27 & 0.65 & \   \\ \hline
pT2               & 8436 & 0.27 & 0.99 & 8551 & 0.27 & 1 & \   \\ \hline
pT4               & 8135 & 0.26 & 0.96 & 8217 & 0.26 & 0.96 & \   \\ \hline
Aplanarity        & 6365 & 0.2 & 0.78 & 6307 & 0.2 & 0.77 & \   \\ \hline
MET/m\_eff(Nj)     & 3900 & 0.12 & 0.61 & 3839 & 0.12 & 0.61 & \   \\ \hline
m\_eff(incl)       & 3715 & 0.12 & 0.95 & 3672 & 0.12 & 0.96 & \   \\ \hline

		\end{tabular}
	\caption{1605.03814 cut flow}
	\label{tab:1605.03814}
\end{table*}

\subsection{arxiv:1704.03848 - Monophoton - ATLAS - 13 TeV}
Brief description of the ATLAS analysis Monophoton at 13 TeV (arxiv:1704.03848).
Results are reported in table~\ref{tab:1704.03848}.

\begin{table*}[htbp]
\tiny
	\centering
		\begin{tabular}{ | l || l | l | l || l | l | l | }
\hline
% Rivet, MadAnalysis 5 and CheckMATE2 benchmarking results & \  & \  & \  & \  & \  & \  & \  & \  \\ \hline
% arxiv:1605.03814 - Jets+MET - ATLAS - 13 TeV & \  & \  & \  & \  & \  & \  & \  & \  \\ \hline
% Gluino mass 1600 - N1 mass 0 & \  & \  & \  & \  & \  & \  & \  & \  \\ \hline
% Rivet default (with efficiencies and smearing); MadAnalysis5 default; CheckMATE2 default & \  & \  & \  & \  & \  & \  & \  & \  \\ \hline
 % & \  & \  & \  & \  & \  & \  & \  & \  \\ \hline
                  &  \multicolumn{3}{c||}{\bf Rivet} & \multicolumn{3}{c||}{\bf MadAnalysis5}    \\ \hline

Description       & \#evt & tot.eff & rel.eff & \#evt & tot.eff & rel.eff    \\ \hline \hline

Initial                    &  	1198	& 1		  & -     & 1198	& 1	 &    -      \\ \hline
ETmiss $>$ 150 GeV           &   	798.3	& 0.67	& 0.67	& 736	& 0.61 &  0.61     \\ \hline
Photon w/ ET $>$ 150 GeV     &   	703.5	& 0.59	& 0.88	& 700	& 0.58 &  0.95     \\ \hline
Pass Tight photon          &   	598.1	& 0.50	& 0.85	& 658	& 0.55 & 	0.94     \\ \hline
Pass Isolated photon       &   	598.1	& 0.50	& 1.00	& 620	& 0.52 & 	0.94     \\ \hline
Pass $\delta\phi$(gamma,MET) $>$ 0.4 &   	597.5	& 0.50	& 1.00	& 596	& 0.50 & 	0.96     \\ \hline
Pass MET/sqrt(SET) $>$ 8.5   &   	538.2	& 0.45	& 0.90	& -	  &  -   &     	     \\ \hline
Pass Jet veto              &   	476.8	& 0.40	& 0.89	& 461	& 0.38 & 	0.77     \\ \hline
Pass Lepton veto           &   	475.5	& 0.40	& 1.00	& 460	& 0.38 & 	1.00     \\ \hline

		\end{tabular}
	\caption{1704.03848 cut flow}
	\label{tab:1704.03848}
\end{table*}


\subsection{CMS-SUS-16-039 - 3 leptons + MET - CMS - 13 TeV}
Brief description of the CMS analysis 3 leptons + MET at 13 TeV (CMS-SUS-16-039).

\subsection{arxiv:1706.04402 - 1 lepton + MET + Jets (>=1b) - CMS - 13 TeV} (topness variable?)
Brief description of the CMS analysis 1 lepton + MET + Jets ($>=1b$) (topness variable) at 13 TeV (arxiv:1706.04402).

\section*{CONCLUSIONS}
We are cool.

\section*{ACKNOWLEDGEMENTS}
We acknowledge the acknowledgements.

\bibliography{sample_bib}

\end{document}
