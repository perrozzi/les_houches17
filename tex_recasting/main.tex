\documentclass[11pt]{cernrep}
\usepackage{graphicx,epsfig, url,cite,slashed}
\usepackage[utf8]{inputenc}
\usepackage[bookmarks,linktocpage]{hyperref}
\bibliographystyle{lesHouches}



\hypersetup{
   colorlinks=true,       % false: boxed links; true: colored links
   linkcolor=blue,        % color of internal links
   citecolor=red,         % color of links to bibliography
   filecolor=magenta,     % color of file links
   }

\begin{document}

\setlength\parindent{0pt}

\title{Recasting activities at LH2017}

\author{
Andy~Buckley$^{AB}$, 
Nishita~Desai$^{ND}$,
Benjamin~Fuks$^{BF1,BF2}$,
Philippe~Gras$^{PG}$, 
David~Grellscheid$^{IPPP}$,
Fabio~Maltoni$^{FM}$,
Olivier~Mattelaer$^{FM}$,
Luca~Perrozzi$^{LP}$,
Peter~Richardson$^{PR1,IPPP}$,
Sezen~Sekmen$^{SS}$
}
% Sabine~Kraml$^{SK}$,
% Gabriel~Facini, 
% Jonathan~Butterworth, 
% Eric~Conte, 
% Pasquale~Musella, 
% Alexandra~Oliveira~Carvalho, 
% Ursula~Laa, 
% Kristin~Lohwasser,
% ???~Thrynova, 
% Efe~Yazgan, 
% Sylvain~Fichet
%}

\institute{
$^{LP}$ IPA at ETH Zurich, Switzerland\\
$^{AB}$ School of Physics \& Astronomy, University of Glasgow, UK\\
$^{BF1}$ Sorbonne Universit\'e, CNRS, Laboratoire de Physique Th\'eorique et Hautes \'Energies, LPTHE, F-75005 Paris, France\\
$^{BF2}$ Institut Universitaire de France, 103 boulevard Saint-Michel, 75005 Paris, France \\
$^{PG}$ IRFU, CEA, Universit\'e Paris-Saclay, Gif-sur-Yvette\\
$^{PR1}$Theoretical Physics Department, CERN, Geneva, Switzerland\\
$^{IPPP}$IPPP, Department of Physics, Durham University, Durham, UK\\
$^{FM}$ CP3/IRMP, Universit\'e Catholique de Louvain, chemin du cyclotron,2 1348 Louvain, Belgique\\
$^{SS}$ Kyungpook National University, Korea
}
% $^2$ \dots\\
%$^{SK}$ Univ. Grenoble Alpes, CNRS, Grenoble INP, LPSC-IN2P3, 38000 Grenoble, France\\

\maketitle

\begin{abstract}
We discuss a first benchmark comparison assessing the performance of different public recasting tools in reproducing ATLAS and CMS analysis results.
\end{abstract}

\section{Introduction}

Searches for new physics constitute a primary objective of the LHC physics program.
Their large number and variety pose severe challenges to both the experimental and theory communities. 
In fact, a plethora of searches in different final states are performed by different physics groups in ATLAS and CMS, 
while new ideas to probe new models and non-trivial signatures and to improve the sensitivity of existing searches constantly emerge.
The ultimate goal of this effort is to discover new physics if such
exists within the reach of the LHC, and to test the widest possible range of hypothetical new physics models.

A typical analysis defines quantities to classify events as signal
or background. They include properties of analysis objects such as
jets, electrons, muons, or global event variables such as object multiplicities,
transverse momenta or transverse masses.
An analysis can be very complex and feature many intricate definitions of object and event
variables, some of which cannot be expressed in closed algebraic form and must be defined
algorithmically. This complexity renders the tasks of visualizing, understanding, developing and
interpreting analyses increasingly challenging.

In the paper publications describing the analyses and their results, the experimental collaborations
provide interpretations of the results in terms of one or more theoretical scenarios the analysis has been designed for. 
Often this is done in the context of so-called simplified models, which consider just a subset of physics states and 
production/decay modes out of a full theory.   
There are, however, a multitude of theories beyond the Standard Model and they come in ever increasing variants. 
To fully assess the implications of the LHC searches for new physics requires the interpretation of the experimental results 
in the context of all these models. This is a very active field with close theory-experiment interaction, 
see e.g.~\cite{reinterpretationforum}, and with several public tools being developed for the (re)interpretation of the 
experimental results.\footnote{This includes also dedicated efforts at Les Houches to provide ''Recommendations for Presentation of LHC Results''~\cite{Brooijmans:2012yi,Kraml:2012sg,Brooijmans:2016vro}.} 
In particular,  
CheckMate~\cite{Drees:2013wra,Cacciari:2005hq},
MadAnalysis~\cite{Conte:2012fm,Conte:2014zja,Dumont:2014tja}
and Rivet~\cite{Waugh:2006ip,Buckley:2010ar} 
aim at reproducing experimental analyses in Monte Carlo simulation, including an approximate emulation of detector effects, 
as new physics searches, which have given only null results so far, are typically not unfolded.  
The scope of this contribution is to provide a first benchmark to compare different public tools in reproducing ATLAS and CMS analysis results.\footnote{It is highly  appreciated that many of these results are provided numerically through HEPDATA~\cite{Maguire:2017ypu} or on the collaboration twiki pages.}

\section{Benchmarking tools and comparison strategy}

The idea behind the exercise described in this section is the implementation of LHC analyses of increasing complexity, in different frameworks followed by a comparison of the results. The exercise is performed with three frameworks, CheckMate~\cite{Drees:2013wra,Cacciari:2005hq}, MadAnalysis~\cite{Conte:2012fm,Conte:2014zja,Dumont:2014tja} and Rivet~\cite{Waugh:2006ip,Buckley:2010ar}, followed by a comparison of the results. We choose two analyses for which a detailed cutflow and detector effects were available. In the future it might be beneficial to use dedicated parsers to convert the analysis described in a common format (denoted LHADA in Fig.~\ref{fig:exercise}) into different recasting codes using for instance the technique described in Contribution~\ref{LHADA}.
Once the analysis are available in the needed format, we attempt to reproduce the new physics
interpretations presented in the original experimental research papers,
validating in this way our reimplementations.
A further step consists in the recasting of the analyses within different new
physics contexts and compare the results among the different frameworks.
A sketch of the recasting exercise workflow is presented in Fig.~\ref{fig:exercise}.

Aside the current scope of the exercise, it is interesting to check how the performance of the Delphes simulation behave across different phase spaces, since they are generally referred to as analysis-spedfic.


% \section{Analysis validation}
\begin{figure}
\begin{center}
\includegraphics[width=0.65\textwidth]{figures/lhada_benchmarking_excersise.png}
 \caption{Sketch of the recasting exercise workflow.
}
\label{fig:exercise}
\end{center}
\end{figure}


\subsection{Analysis frameworks and tools}
In this section we describe the analysis frameworks and tools used for the comparison and benchmarking

\subsubsection{CheckMate}
CheckMATE~\cite{Drees:2013wra,Cacciari:2005hq} takes simulated event files in .hep or .hepmc for any model as input and simply returns if the underlying model is `excluded' or `allowed' after
performing a detector simulation and testing various implemented analyses. The embedded AnalysisManager allows for the embedding of additional current and
prospective future LHC results from ATLAS and CMS which have not yet been implemented.
Detector effects are modeled by Delphes with a tune containing efficiency functions for lepton reconstruction and flavour tagging. The soon-to-be published version 2.0 of the code adds the possibility of using Pythia 8~\cite{Sjostrand:2007gs} to
generate supersymmetric events on-the-fly or to shower provided Les Houches
event files for any model. Currently, the collaboration is working on an extension to enable the on-the-fly simulation of events for any model.


\subsubsection{MadAnalysis}
MadAnalysis 5~\cite{Conte:2012fm,Conte:2014zja} is a generic user-friendly
framework for phenomenological investigations at particle colliders, {\it i.e.}
to perform physics analyses of Monte Carlo event files. While prospective
analyses of hard scattering events, parton showered events, hadronized events or
reconstructed events can be designed easily thanks to its Python-based
meta-language, MadAnalysis also allows for the recasting of LHC analyses on new
physics signals provided under the form of .hep and .hepmc event files. The
output here consists in the confidence level at which the model signals are
excluded.
Its Public Analysis Database~\cite{Dumont:2014tja} comprises a growing
collection of LHC analyses which have been implemented in the MadAnalysis 5
framework for the purpose of recasting. Delphes is used for the detector
simulation. For each implemented analysis, a detailed validation note is
provided and the public analysis database follows an open-source policy. Only
contributed codes provided with a detailed validation note are published, and
they are moreover citable via Inspire. The framework being integrated within
MadGraph5\_aMC@NLO~\cite{Alwall:2014hca}, it provides a full recast chain linking a model
and its associated signatures to limit setting.

\subsubsection{Rivet}
Originally developed as a toolkit for the validation of Monte Carlo event
generators, Rivet~\cite{Waugh:2006ip,Buckley:2010ar} (Robust Independent Validation of Experiment and
Theory) has become a standard for documenting (unfolded) Standard Model (SM)
measurements. The top and Higgs physics working groups of all LHC experiments
are increasingly providing Rivet routines for their analyses. Rivet analyses are
written in a user-friendly subset of C++11, and are picked up at runtime as
`plugin libraries'; they can be executed on an event stream either through a Python script interface, or by direct code interfacing to a C++ API.
The original SM-focused requirement of unfolded observables made Rivet
inappropriate for beyond the Standard Model (BSM) searches (other than those
using just jets and missing energy) until the addition of
detector-smearing/efficiency machinery in Rivet~2.5.0. This detector machinery provides equivalent efficiency effects to a Delphes-type simulation, and imitates the less important kinematic smearing of physics objects to within a few percent. A novel feature is that the Rivet detector implementation allows for using
different jet algorithms, lepton and b-tagging operating points, full-detailed object isolation algorithms, and resolutions/efficiencies specific to each
analysis procedure and event selection. This hence allows for a more accurate detector modelling and more robust analysis preservation than `global' detector simulations in addressing some experiment requests for `official fast-sim' tools. The aim is to encourage Rivet code provision directly from BSM data analysers, as is already the case for SM results: additional tools to assist BSM analysis implementation are being added on request.

\section{Analyses benchmarking, comparisons and results}

\subsection{An ATLAS search for supersymmetry in a final
state with jets and missing energy (13 TeV, 3.2~fb$^{-1}$)}

In the analysis of Ref.~\cite{Aaboud:2016zdn}, the ATLAS collaboration targets
the production of the strongly-interacting superpartners of the Standard Model
QCD partons, followed by their decay into jets and missing energy carried by
neutralinos. 3.2~fb$^{-1}$ of proton-proton LHC collisions at a center-of-mass
energy of 13~TeV are analyzed.

The analysis focuses on jets reconstructed by means of the anti-$k_T$
algorithm~\cite{Cacciari:2008gp} with a radius parameter set to $R=0.4$, with
a transverse momentum larger than 20~GeV and a pseudorapidity $|\eta|<2.8$.
Events featuring loosely reconstructed electrons and muons are vetoed.
Event preselection requires a significant amount of missing energy,
$\slashed{E}_T > 200$~GeV and the transverse-momentum of the leading jet is
imposed to be larger than 200~GeV and 300~GeV if two or more than two jets are
reconstructed, respectively.

\begin{table*}[b!]
\footnotesize
 \centering
  \begin{tabular}{ | l || l | l | l || l | l | l || l | }
\hline
    &  \multicolumn{3}{c||}{\bf Rivet} & \multicolumn{3}{c||}{\bf MadAnalysis 5} &   {\bf CheckMATE}   \\ \hline
  Description       & \#evt & tot.eff & rel.eff & \#evt & tot.eff & rel.eff &   tot.eff   \\ \hline \hline
{\bf 2jl cut-flow}                  & 31250 & 1 & - & 31250 & 1 & - & \   \\ \hline
Pre-sel+MET+pT1   & 28592 & 0.91 & 0.91 & 28626 & 0.92 & 0.92 & \   \\ \hline
Njet              & 28592 & 0.91 & 1 & 28625 & 0.92 & 1 & \   \\ \hline
Dphi\_min(j,MET)   & 17297 & 0.55 & 0.6 & 17301 & 0.55 & 0.6 & \   \\ \hline
pT2               & 17067 & 0.55 & 0.99 & 17042 & 0.55 & 0.99 & \   \\ \hline
MET/sqrtHT        & 8900 & 0.28 & 0.52 & 8898 & 0.28 & 0.52 & \   \\ \hline
m\_eff(incl)       & 8896 & 0.28 & 1 & 8897 & 0.28 & 1 & \   \\ \hline
\hline
{\bf 2jm cut-flow} & 31250 & 1 & - & 32150 & 1 & - & 1  \\ \hline
Pre-sel+MET+pT1   & 28472 & 0.91 & 0.91 & 28478 & 0.91 & 0.91 & 0.91  \\ \hline
Njet              & 28472 & 0.91 & 1 & 28477 & 0.91 & 1 & 0.91  \\ \hline
Dphi\_min(j,MET)   & 22950 & 0.73 & 0.81 & 22889 & 0.73 & 0.8 & 0.73  \\ \hline
pT2               & 22950 & 0.73 & 1 & 22889 & 0.73 & 1 & 0.73  \\ \hline
MET/sqrtHT        & 10730 & 0.34 & 0.47 & 10710 & 0.34 & 0.47 & 0.33  \\ \hline
m\_eff(incl)       & 10630 & 0.34 & 0.99 & 10609 & 0.34 & 0.99 & 0.32  \\ \hline
\hline
{\bf 2jt cut-flow} & 31250 & 1 & - & 31250 & 1 & - & \   \\ \hline
Pre-sel+MET+pT1   & 28592 & 0.91 & 0.91 & 28626 & 0.92 & 0.92 & \   \\ \hline
Njet              & 28592 & 0.91 & 1 & 28625 & 0.92 & 1 & \   \\ \hline
Dphi\_min(j,MET)   & 17297 & 0.55 & 0.6 & 17301 & 0.55 & 0.6 & \   \\ \hline
pT2               & 17067 & 0.55 & 0.99 & 17042 & 0.55 & 0.99 & \   \\ \hline
MET/sqrtHT        & 5083 & 0.16 & 0.3 & 5098 & 0.16 & 0.3 & \   \\ \hline
Pass m\_eff(incl)  & 4861 & 0.16 & 0.96 & 4889 & 0.16 & 0.96 & \   \\ \hline
		\end{tabular}
 \caption{Number of events surviving each selection, total and relative
  selection efficiencies as obtained with Rivet and MadAnalysis 5 for the dijet
  signal regions of the multijet+missing energy ATLAS analysis of
  Ref.~\cite{Aaboud:2016zdn}. Partly available Checkmate results for the total
  efficiencies are also indicated.}
	\label{tab:1605.03814-2j}
\end{table*}


\begin{table*}
\begin{tabular}{ | l || l | l | l || l | l | l || l | }
\hline
    &  \multicolumn{3}{c||}{\bf Rivet} & \multicolumn{3}{c||}{\bf MadAnalysis 5} &   {\bf CheckMATE}   \\ \hline
  Description       & \#evt & tot.eff & rel.eff & \#evt & tot.eff & rel.eff &   tot.eff   \\ \hline \hline
\hline
{\bf 4jt cut-flow} & 31250 & 1 & - & 31250 & 1 & - & 1  \\ \hline
Pre-sel+MET+pT1   & 28592 & 0.91 & 0.91 & 28626 & 0.92 & 0.92 & 0.91  \\ \hline
Njet              & 27322 & 0.87 & 0.96 & 27128 & 0.87 & 0.95 & 0.87  \\ \hline
Dphi\_min(j,MET)   & 18929 & 0.61 & 0.69 & 18829 & 0.6 & 0.69 & 0.6  \\ \hline
pT2               & 18715 & 0.6 & 0.99 & 18825 & 0.6 & 1 &     --       \\ \hline
pT4               & 16610 & 0.53 & 0.89 & 16430 & 0.53 & 0.87 & 0.52  \\ \hline
Aplanarity        & 11849 & 0.38 & 0.71 & 11395 & 0.36 & 0.69 & 0.36  \\ \hline
MET/m\_eff(Nj)     & 8334 & 0.27 & 0.7 & 7971 & 0.26 & 0.7 & 0.25  \\ \hline
m\_eff(incl)       & 7201 & 0.23 & 0.86 & 6972 & 0.22 & 0.87 & 0.21  \\ \hline
\hline
{\bf 5j cut-flow} & 31250 & 1 & - & 31250 & 1 & - & 1 \\ \hline
Pre-sel+MET+pT1   & 28592 & 0.91 & 0.91 & 28626 & 0.92 & 0.92 & 0.91 \\ \hline
Njet              & 21234 & 0.68 & 0.74 & 21185 & 0.68 & 0.74 & 0.68 \\ \hline
Dphi\_min(j,MET)   & 14294 & 0.46 & 0.67 & 14292 & 0.46 & 0.67 & 0.45 \\ \hline
pT2               & 14146 & 0.45 & 0.99 & 14289 & 0.46 & 1 &    --       \\ \hline
pT4               & 13229 & 0.42 & 0.94 & 13228 & 0.42 & 0.93 & 0.42 \\ \hline
Aplanarity        & 9836 & 0.31 & 0.74 & 9576 & 0.31 & 0.72 & 0.3 \\ \hline
MET/m\_eff(Nj)     & 4643 & 0.15 & 0.47 & 4506 & 0.14 & 0.47 & 0.13 \\ \hline
m\_eff(incl)       & 4620 & 0.15 & 1 & 4476 & 0.14 & 0.99 & 0.13 \\ \hline
\hline
{\bf 6jm cut-flow} & 31250 & 1 & - & 31250 & 1 & - & 1  \\ \hline
Pre-sel+MET+pT1   & 28592 & 0.91 & 0.91 & 28626 & 0.92 & 0.92 & 0.91  \\ \hline
Njet              & 13235 & 0.42 & 0.46 & 13236 & 0.42 & 0.46 & 0.41  \\ \hline
Dphi\_min(j,MET)   & 8520 & 0.27 & 0.64 & 8553 & 0.27 & 0.65 & 0.26  \\ \hline
pT2               & 8436 & 0.27 & 0.99 & 8551 & 0.27 & 1 &    --        \\ \hline
pT4               & 8135 & 0.26 & 0.96 & 8217 & 0.26 & 0.96 & 0.25  \\ \hline
Aplanarity        & 6365 & 0.2 & 0.78 & 6307 & 0.2 & 0.77 & 0.19  \\ \hline
MET/m\_eff(Nj)     & 2675 & 0.09 & 0.42 & 2665 & 0.09 & 0.42 & 0.08  \\ \hline
m\_eff(incl)       & 2670 & 0.09 & 1 & 2656 & 0.08 & 1 & 0.08  \\ \hline
\hline
{\bf 6jt cut-flow} & 31250 & 1 & - & 31250 & 1 & - & \   \\ \hline
Pre-sel+MET+pT1   & 28592 & 0.91 & 0.91 & 28626 & 0.92 & 0.92 & \   \\ \hline
Njet              & 13235 & 0.42 & 0.46 & 13236 & 0.42 & 0.46 & \   \\ \hline
Dphi\_min(j,MET)   & 8520 & 0.27 & 0.64 & 8553 & 0.27 & 0.65 & \   \\ \hline
pT2               & 8436 & 0.27 & 0.99 & 8551 & 0.27 & 1 & \   \\ \hline
pT4               & 8135 & 0.26 & 0.96 & 8217 & 0.26 & 0.96 & \   \\ \hline
Aplanarity        & 6365 & 0.2 & 0.78 & 6307 & 0.2 & 0.77 & \   \\ \hline
MET/m\_eff(Nj)     & 3900 & 0.12 & 0.61 & 3839 & 0.12 & 0.61 & \   \\ \hline
m\_eff(incl)       & 3715 & 0.12 & 0.95 & 3672 & 0.12 & 0.96 & \   \\ \hline
 \end{tabular}
 \caption{Same as in Table~\ref{tab:1605.03814-2j} but for the signal regions
  targeting final states containing four, five and six jets.}
	\label{tab:1605.03814-nj}
\end{table*}

The analysis is then divided into seven signal regions focusing on different jet
multiplicities (from 2 to 6) with different transverse-momentum thresholds. The
missing transverse momentum is then enforced to be well separated from the
leading reconstructed jets, and its significance is constrained for events
featuring only two jets. For cases where at least four jets are reconstructed,
additional selections on the aplanarity variable
% ~\cite{Chen:2011ah} 
and the effective mass, {\it i.e} the scalar sum of the transverse momenta of the
reconstructed and the missing transverse energy.

Implementations of this analysis are available in Checkmate, MadAnalysis\,5 (recast code \cite{ma5-multijet}) 
and Rivet (ATLAS\_2016\_I1458270~\cite{rivet-multijet}). 

We generated signal events for a gluino pair production in the simplified model considered in Ref~\cite{Aaboud:2016zdn} with a direct decay of the gluino into SM particles and the lighest supersymmetric particle (LSP). The gluino mass is set to 1.6$\,$TeV mass and the LSP is assumed to be massless. The pseudo-data samples have been generated by using MadGraph5\_aMC@NLO~\cite{Alwall:2014hca} and Pythia8~\cite{Sjostrand:2014zea}.

The comparison of
predictions for the cutflows as obtained with MadAnalysis\,5  and Rivet are
reported in Table~\ref{tab:1605.03814-2j} and Table~\ref{tab:1605.03814-nj} for
all seven signal regions.  The tables include the total number of events
surviving each selection, the associated cut efficiency and the total efficiency
evaluated with respect to the initial number of events. Partially available
Checkmate results are also indicated for what concern the total efficiencies and
for a few signal regions.
An excellent agreement between the three codes has been obtained.





\subsection{An ATLAS search for dark matter in the monophoton final state
  (13~TeV,  36.1~fb$^{-1}$)}

In the analysis of Ref.~\cite{Aaboud:2017dor}, the ATLAS collaboration has
searched for dark matter when it is produced in association with a very
energetic photon. The search results have been reinterpreted in dark matter
simplified scenarios in which a pair of dark matter particles is produced in
association with a photon arising from initial state radiation. 36.1~fb$^{-1}$
of proton-proton LHC collisions at a center-of-mass energy have been analyzed.

The analysis requires the presence of at least one tightly-isolated photon with
a transverse energy $E_T > 150$~GeV and with a pseudorapidity satisfying
$|\eta| < 2.37$, the pseudorapidity region $1.37 < |\eta| < 2.37$ being
excluded. Events featuring loose eletrons and muons and more than one jets with
a transverse momentum larger than 30~GeV and a pseudorapidity $|\eta|<4.5$ are
vetoed. As in the previous analysis, jets are reconstructed by means of the
anti-$k_T$ algorithm~\cite{Cacciari:2008gp}  and a radius parameter set to
$R=0.4$. In addition, event selection requires a missing transverse energy
significance larger than 8.5~GeV$^{1/2}$, and the missing transverse momentum
has top be well separated from the photon and the jet (for events featuring one
reconstructed jet).

Five signal regions are defined according to different requirements on the
amount of missing transverse energy, namely three inclusive regions and two
non-overlapping exclusive regions.

We generated events using the simplified model of dark matter (DM) production involving an axial-vector operator, Dirac DM and couplings g$_{\mathrm{q}}= 0.25$ and g$_{\chi} = 1$ with m$_{\chi} = 10\,$GeV and m$_{\mathrm{med}}=800\,$GeV described in Ref.~\cite{Aaboud:2017dor}.

In Table~\ref{tab:1704.03848}, we compare the total number of events surviving each selection, the associated cut
efficiency and the total efficiency evaluated with respect to the initial number
of events as obtained with MadAnalysis5 (recast code~\cite{ma5-monophoton}) 
and Rivet %SK: give reference of analysis code.
. Whilst a fair agreement is
obtained between two codes, differences of 5\%--10\% are observed for a few 
cuts. This can be traced back to the missing energy modelling that is complicated
to reproduce. 
The final acceptances of about 40\% (Rivet) and 38\% (MadAnalysis) are however 
in good agreement.  


\begin{table*}
 \centering
  \begin{tabular}{ | l || l | l | l || l | l | l | }
\hline
                  &  \multicolumn{3}{c||}{\bf Rivet} & \multicolumn{3}{c||}{\bf MadAnalysis 5}    \\ \hline

Description       & \#evt & tot.eff & rel.eff & \#evt & tot.eff & rel.eff    \\ \hline \hline

Initial                    &  	1198	& 1		  & -     & 1198	& 1	 &    -      \\ \hline
ETmiss $>$ 150 GeV           &   	798.3	& 0.67	& 0.67	& 736	& 0.61 &  0.61     \\ \hline
Photon w/ ET $>$ 150 GeV     &   	703.5	& 0.59	& 0.88	& 700	& 0.58 &  0.95     \\ \hline
Pass Tight photon          &   	598.1	& 0.50	& 0.85	& 658	& 0.55 & 	0.94     \\ \hline
Pass Isolated photon       &   	598.1	& 0.50	& 1.00	& 620	& 0.52 & 	0.94     \\ \hline
Pass $\delta\phi$(gamma,MET) $>$ 0.4 &   	597.5	& 0.50	& 1.00	& 596	& 0.50 & 	0.96     \\ \hline
Pass MET/sqrt(SET) $>$ 8.5   &   	538.2	& 0.45	& 0.90	& -	  &  -   &     	     \\ \hline
Pass Jet veto              &   	476.8	& 0.40	& 0.89	& 461	& 0.38 & 	0.77     \\ \hline
Pass Lepton veto           &   	475.5	& 0.40	& 1.00	& 460	& 0.38 & 	1.00     \\ \hline
  \end{tabular}
 \caption{Number of events surviving each selection, total and relative
  selection efficiencies as obtained with Rivet and MadAnalysis 5 for the SRI1
  signal region of the monophoton ATLAS analysis of Ref.~\cite{Aaboud:2017dor}.}
 \label{tab:1704.03848}
\end{table*}

%\clearpage

\section*{CONCLUSIONS}
We presented a first benchmark comparison of the performance of different recasting tools 
which reproduce LHC analyses in Monte Carlo simulation.
For the two cases treated here, good agreement is found between the different frameworks and detector simulation techniques. 
The comparison is ongoing with several more analyses which are currently being validated. 
It will also be interesting to compare performances for different signal scenarios, to assess the reliability of the 
recasting methods in, e.g.\  extreme regions of phase space and/or for very different signal hypotheses the the 
one the analyses have been designed for.

\section*{ACKNOWLEDGEMENTS}
The authors would like to thank Sabine~Kraml for the fruitful discussions and the help in finalising the contribution.


\bibliography{sample_bib}

\end{document}
